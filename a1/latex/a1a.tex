%%%%%%%%%%%%%%%%%%%%%%%%%%%%%%%%%%%%%%%%%
% Short Sectioned Assignment
% LaTeX Template
% Version 1.0 (5/5/12)
%
% This template has been downloaded from:
% http://www.LaTeXTemplates.com
%
% Original author:
% Frits Wenneker (http://www.howtotex.com)
%
% License:
% CC BY-NC-SA 3.0 (http://creativecommons.org/licenses/by-nc-sa/3.0/)
%
%%%%%%%%%%%%%%%%%%%%%%%%%%%%%%%%%%%%%%%%%

%----------------------------------------------------------------------------------------
%	PACKAGES AND OTHER DOCUMENT CONFIGURATIONS
%----------------------------------------------------------------------------------------

\documentclass{article} % A4 paper and 11pt font size

\usepackage[T1]{fontenc} % Use 8-bit encoding that has 256 glyphs
\usepackage[english]{babel} % English language/hyphenation
\usepackage{amsmath} % Math packages
\usepackage{enumerate}
\usepackage{algpseudocode}

\usepackage{fancyhdr} % Custom headers and footers
\pagestyle{fancyplain} % Makes all pages in the document conform to the custom headers and footers
\fancyhead{} % No page header - if you want one, create it in the same way as the footers below
\fancyfoot[L]{} % Empty left footer
\fancyfoot[C]{} % Empty center footer
\fancyfoot[R]{\thepage} % Page numbering for right footer
\setlength{\headheight}{13.6pt} % Customize the height of the header
\renewcommand{\headrulewidth}{0pt} % Remove header underlines
\renewcommand{\footrulewidth}{0pt} % Remove footer underlines

\allowdisplaybreaks

% Margins
\topmargin=-0.8in
\evensidemargin=0in
\oddsidemargin=0in
\textwidth=6.5in
\textheight=9.0in
\headsep=0.5in

%----------------------------------------------------------------------------------------
%	TITLE SECTION
%----------------------------------------------------------------------------------------

\newcommand{\horrule}[1]{\rule{\linewidth}{#1}} % Create horizontal rule command with 1 argument of height

\title{
\normalfont \normalsize
\textsc{University of Toronto} \\ [25pt] % Your university, school and/or department name(s)
\horrule{0.5pt} \\[0.4cm] % Thin top horizontal rule
\huge Assignment Title \\ % The assignment title
\horrule{2pt} \\[0.5cm] % Thick bottom horizontal rule
}

\author{Nicholas Dujay\\999194900} % Your name

\date{\normalsize\today} % Today's date or a custom date

\begin{document}

\maketitle % Print the title

%----------------------------------------------------------------------------------------
%	PROBLEM 1
%----------------------------------------------------------------------------------------

\section{Question 1}

\begin{enumerate}[a)]

%----------------------------------------------------------------------------------------
% part a
%----------------------------------------------------------------------------------------
\item

The tangent vector is the derivative of the parametric form with respect to t.

Tangent vector:
\begin{align*}
\frac{\delta p(t)}{\delta t} &= \left(\frac{\delta x(t)}{\delta t}, \frac{\delta y(t)}{\delta t}\right)\\
&= \left(\frac{\delta (at)}{\delta t}, \frac{\delta (-\frac{1}{2}gt^2 + bt + h)}{\delta t}\right)\\
&= (a, b - gt)
\end{align*}

The normal vector is just any vector that is perpendicular to the tangent vector. $\frac{\delta p(t)}{\delta t} \cdot \vec{n} = 0$, $\vec{n} = (b - gt, -a)$.

%----------------------------------------------------------------------------------------

%----------------------------------------------------------------------------------------
% part b
%----------------------------------------------------------------------------------------
\item

To find the time of impact $t_i$, we must solve the quadratic equation for $y(t) = 0$.
\begin{align*}
y(t_i) &= 0\\
-\frac{1}{2}gt_i^2 + bt_i + h &= 0\\
t_i &= \frac{-b \pm \sqrt{b^2 - 4(-\frac{1}{2}g)(h)} }{2(-\frac{1}{2}g)}\\
t_i &= \frac{b \mp \sqrt{b^2 + 2gh} }{g}\\
\end{align*}
Since $\sqrt{b^2 + 2gh}$ is always positive, we choose $t_i = \frac{b + \sqrt{b^2 + 2gh} }{g}$.

The velocity at $t_i$:
\begin{align*}
\frac{\delta p(t_i)}{\delta t} &= (a, b - gt_i)\\
&= \left(a, b - g\left(\frac{b + \sqrt{b^2 + 2gh} }{g}\right)\right)\\
&= \left(a, -\sqrt{b^2 + 2gh}\right)
\end{align*}

The location at $t_i$:
\begin{align*}
p(t_i) &= (x(t_i), y(t_i))\\
&= \left(\frac{a(b + \sqrt{b^2 + 2gh}) }{g}, 0\right)
\end{align*}
%----------------------------------------------------------------------------------------

\end{enumerate}

%----------------------------------------------------------------------------------------

%----------------------------------------------------------------------------------------
%	PROBLEM 2
%----------------------------------------------------------------------------------------

\section{Question 2}

\begin{enumerate}[a)]

%----------------------------------------------------------------------------------------
% part a
%----------------------------------------------------------------------------------------
\item

\begin{align*}
\text{(Translate in x and then shear in x)} &=
\left[
\begin{matrix}
1 & s_x & 0\\
0 & 1 & 0\\
0 & 0 & 1
\end{matrix}
\right]
\cdot
\left[
\begin{matrix}
1 & 0 & t_x\\
0 & 1 & 0\\
0 & 0 & 1
\end{matrix}
\right]\\
&= \left[
\begin{matrix}
1 & s_x & t_x\\
0 & 1 & 0\\
0 & 0 & 1
\end{matrix}
\right]\end{align*}

\begin{align*}
\text{(Shear in x and then translate in x)} &=
\left[
\begin{matrix}
1 & 0 & t_x\\
0 & 1 & 0\\
0 & 0 & 1
\end{matrix}
\right]
\cdot
\left[
\begin{matrix}
1 & s_x & 0\\
0 & 1 & 0\\
0 & 0 & 1
\end{matrix}
\right]\\
&= \left[
\begin{matrix}
1 & s_x & t_x\\
0 & 1 & 0\\
0 & 0 & 1
\end{matrix}
\right]\\
&= \text{(Translate in x and then shear in x )}
\end{align*}
As you can see, translation and shearing in x is commutative.
%----------------------------------------------------------------------------------------

%----------------------------------------------------------------------------------------
% part b
%----------------------------------------------------------------------------------------
\item

\begin{align*}
\text{(Rotate by $\phi$ and then by $\theta$)} &=
\left[
\begin{matrix}
cos(\theta) & -sin(\theta)\\
sin(\theta) & cos(\theta)
\end{matrix}
\right]
\cdot
\left[
\begin{matrix}
cos(\phi) & -sin(\phi)\\
sin(\phi) & cos(\phi)
\end{matrix}
\right]\\
&= \left[
\begin{matrix}
cos(\theta) cos(\phi) - sin(\theta) sin(\phi) & -cos(\theta) sin(\phi) - sin(\theta) cos(\phi)\\
sin(\theta) cos(\phi) + cos(\theta) sin(\phi) & -sin(\theta) sin(\phi) + cos(\theta) cos(\phi)
\end{matrix}
\right]
\end{align*}


\begin{align*}
\text{(Rotate by $\theta$ and then by $\phi$)} &=
\left[
\begin{matrix}
cos(\phi) & -sin(\phi)\\
sin(\phi) & cos(\phi)
\end{matrix}
\right]
\cdot
\left[
\begin{matrix}
cos(\theta) & -sin(\theta)\\
sin(\theta) & cos(\theta)
\end{matrix}
\right]\\
&= \left[
\begin{matrix}
cos(\phi) cos(\theta) - sin(\phi) sin(\theta) & -cos(\phi) sin(\theta) - sin(\phi) cos(\theta)\\
sin(\phi) cos(\theta) + cos(\phi) sin(\theta) & -sin(\phi) sin(\theta) + cos(\phi) cos(\theta)
\end{matrix}
\right]\\
&= \left[
\begin{matrix}
cos(\theta) cos(\phi) - sin(\theta) sin(\phi) & -cos(\theta) sin(\phi) - sin(\theta) cos(\phi)\\
sin(\theta) cos(\phi) + cos(\theta) sin(\phi) & -sin(\theta) sin(\phi) + cos(\theta) cos(\phi)
\end{matrix}
\right]\\
&=
\left[
\begin{matrix}
cos(\theta) & -sin(\theta)\\
sin(\theta) & cos(\theta)
\end{matrix}
\right]
\cdot
\left[
\begin{matrix}
cos(\phi) & -sin(\phi)\\
sin(\phi) & cos(\phi)
\end{matrix}
\right]\\
&= \text{(Rotate by $\phi$, and then by $\theta$.)}
\end{align*}
As you can see, two rotations are commutative.
%----------------------------------------------------------------------------------------
% part c
%----------------------------------------------------------------------------------------
\item

\begin{align*}
\text{(Rotate by $\phi$ and then scale by s)} &=
\left[
\begin{matrix}
s & 0\\
0 & s
\end{matrix}
\right]
\cdot
\left[
\begin{matrix}
cos(\phi) & -sin(\phi)\\
sin(\phi) & cos(\phi)
\end{matrix}
\right]\\
&= \left[
\begin{matrix}
s \cdot cos(\phi)  & -s \cdot sin(\phi)\\
s \cdot sin(\phi) & s \cdot cos(\phi)
\end{matrix}
\right]
\end{align*}

\begin{align*}
\text{(Scale by s and then rotate by $\phi$)} &=
\left[
\begin{matrix}
cos(\phi) & -sin(\phi)\\
sin(\phi) & cos(\phi)
\end{matrix}
\right]
\cdot
\left[
\begin{matrix}
s & 0\\
0 & s
\end{matrix}
\right]
\\
&= \left[
\begin{matrix}
s \cdot cos(\phi)  & -s \cdot sin(\phi)\\
s \cdot sin(\phi) & s \cdot cos(\phi)
\end{matrix}
\right]\\
&= \text{(Rotate by $\phi$ and then scale by s)}
\end{align*}
As you can see, uniform scaling and rotation are commutative.
%----------------------------------------------------------------------------------------

%----------------------------------------------------------------------------------------
% part d
%----------------------------------------------------------------------------------------
\item

\begin{align*}
\text{(Rotate by $\phi$ and then scale by $s_x, s_y$)} &=
\left[
\begin{matrix}
s_x & 0\\
0 & s_y
\end{matrix}
\right]
\cdot
\left[
\begin{matrix}
cos(\phi) & -sin(\phi)\\
sin(\phi) & cos(\phi)
\end{matrix}
\right]\\
&= \left[
\begin{matrix}
s_x \cdot cos(\phi)  & -s_x \cdot sin(\phi)\\
s_y \cdot sin(\phi) & s_y \cdot cos(\phi)
\end{matrix}
\right]
\end{align*}

\begin{align*}
\text{(Scale by $s_x, s_y$ and then rotate by $\phi$)} &=
\left[
\begin{matrix}
cos(\phi) & -sin(\phi)\\
sin(\phi) & cos(\phi)
\end{matrix}
\right]
\cdot
\left[
\begin{matrix}
s_x & 0\\
0 & s_y
\end{matrix}
\right]
\\
&= \left[
\begin{matrix}
s_x \cdot cos(\phi)  & -s_y \cdot sin(\phi)\\
s_x \cdot sin(\phi) & s_y \cdot cos(\phi)
\end{matrix}
\right]\\
&\neq \text{(Rotate by $\phi$ and then scale by s)}
\end{align*}
As you can see, non uniform scaling and rotation are \textbf{not} commutative.
%----------------------------------------------------------------------------------------

%----------------------------------------------------------------------------------------

\end{enumerate}

%----------------------------------------------------------------------------------------

%----------------------------------------------------------------------------------------
%	PROBLEM 3
%----------------------------------------------------------------------------------------

\section{Question 3}

\begin{enumerate}[a)]

%----------------------------------------------------------------------------------------
% part a
%----------------------------------------------------------------------------------------
\item

First, we get the direction vector from $\vec{p}_i$ to $\vec{p}_{i+1}$: $(x_{i+1} - x_i, y_{i+1} - y_i)$. Then, since the inward facing normal is a 90 degree counter clockwise rotation of this vector, the result is $(y_i - y_{i+1}, x_{i+1} - x_i)$.

Counter clockwise rotation:
\begin{align*}
\left[
\begin{matrix}
cos(90) & -sin(90)\\
sin(90) & cos(90)
\end{matrix}
\right]
\cdot
\left[
\begin{matrix}
x\\
y
\end{matrix}
\right] = \left[
\begin{matrix}
0 & -1\\
1 & 0
\end{matrix}
\right]
\cdot
\left[
\begin{matrix}
x\\
y
\end{matrix}
\right] = \left[
\begin{matrix}
-y\\
x
\end{matrix}
\right]
\end{align*}
%----------------------------------------------------------------------------------------

%----------------------------------------------------------------------------------------
% part b
%----------------------------------------------------------------------------------------
\item
Let $d_q$ be a vector from $q$ to $p_i$, ie $d_q = p_i - q$. The following function tells you if the point is on the inward facing line:
\[
sameside(q, p_i, n_i) =
\begin{cases}
true & \text{if } d_q \cdot n_i > 0\\
false & \text{if } d_q \cdot n_i \leq 0
\end{cases}
\]

If the angle between $d_q$ and $n_i$ is less than 90 degrees, we know $q$ is on the inward side. Since $\forall \theta \in [0,90) \cdot cos(\theta) > 0$, then the dot product between $d_q$ and $n_i$ is positive if $q$ is on the inward side, and negative if $q$ is on the outward side.
%----------------------------------------------------------------------------------------

%----------------------------------------------------------------------------------------
% part c
%----------------------------------------------------------------------------------------
\item

The algorithm will use the above two parts to the question to solve this. The high level steps are the following:

\begin{enumerate}[i)]
\item For each pair of points $(p_i, p_{i+1})$ calculate the \textbf{inward facing normal} $n_{p_i}$.
\item For each pair of points $(r_i, r_{i+1})$ calculate the \textbf{outward facing normal} $n_{r_i}$. Note: in the actual implementation, we can just reverse the order of points $(r_{i+1}, r_i)$ and calculate the inward facing normal.
\item For all of the above normals, ensure that \textbf{sameside} (from part b) is true for the point.
\end{enumerate}

Some pseudo code:
\begin{algorithmic}
\Function{normal}{$p_{i+1}$, $p_i$}
\State \Return ($y_i - y_{i+1}$, $x_{i+1} - x_i$)
\EndFunction
\State
\Function{InConvexPolgyon}{point}
\ForAll{$p_i, p_{i+1}$ in the outer polygon} \Comment{check if the point is inside the outer polygon}
\If{not sameside(point, $p_i$, \Call{Normal}{$p_{i+1}$,$p_i$})}
\State \Return False
\EndIf
\EndFor

\ForAll{$r_j, r_{j+1}$ in the inner polygon} \Comment{check if the point is outside the inner polygon}
\If{not sameside(point, $r_i$, \Call{Normal}{$r_j$,$r_{j+1}$})}
\State \Return False \Comment{reversing the order gives you an outward facing normal here}
\EndIf
\EndFor
\State \Return True 
\EndFunction
\end{algorithmic}

%----------------------------------------------------------------------------------------

\end{enumerate}

%----------------------------------------------------------------------------------------

%----------------------------------------------------------------------------------------
%	PROBLEM 4
%----------------------------------------------------------------------------------------

\section{Question 4}

%----------------------------------------------------------------------------------------

%----------------------------------------------------------------------------------------
%	PROBLEM 5
%----------------------------------------------------------------------------------------

\section{Question 5}

First, lets express the transformation as a translation and then followed by a scale and finally followed by a rotation. Let the given matrix $ \left[
\begin{matrix}
8 & 3 & -7\\
6 & -4 & -24\\
0 & 0 & 1
\end{matrix}
\right]$ be $A$,
Let $R(\theta)$ be the rotation matrix, $S(s_x, s_y)$ be the non uniform scale matrix, $T(t_x, t_y)$ be the translation matrix.
\begin{align*}
R(\theta) \cdot S(s_x, s_y) \cdot T(t_x, t_y) &=
R(\theta) \cdot S(s_x, s_y) \cdot
\left[
\begin{matrix}
1 & 0 & t_x\\
0 & 1 & t_y\\
0 & 0 & 1
\end{matrix}
\right]\\
&=
R(\theta) \cdot
\left[
\begin{matrix}
s_x & 0 & 0\\
0 & s_y & 0\\
0 & 0 & 1
\end{matrix}
\right]
\cdot
\left[
\begin{matrix}
1 & 0 & t_x\\
0 & 1 & t_y\\
0 & 0 & 1
\end{matrix}
\right]\\
&=
R(\theta) \cdot
\left[
\begin{matrix}
s_x & 0 & s_x t_x\\
0 & s_y & s_y t_y\\
0 & 0 & 1
\end{matrix}
\right]\\
&= \left[
\begin{matrix}
cos(\theta) & -sin(\theta) & 0\\
sin(\theta) & cos(\theta) & 0\\
0 & 0 & 1
\end{matrix}
\right]
\cdot
\left[
\begin{matrix}
s_x & 0 & s_x t_x\\
0 & s_y & s_y t_y\\
0 & 0 & 1
\end{matrix}
\right]\\
&= \left[
\begin{matrix}
s_x cos(\theta) & - s_y sin(\theta) & s_x cos(\theta) t_x - s_y sin(\theta) t_y\\
s_x sin(\theta) & s_y cos(\theta) & s_x sin(\theta) t_x + s_y cos(\theta) t_y\\
0 & 0 & 1
\end{matrix}
\right]
\end{align*}

Setting $A = R(\theta) \cdot S(s_x, s_y) \cdot T(t_x, t_y)$ will give us the following system of equations.
\begin{align*}
\left[
\begin{matrix}
8 & 3 & -7\\
6 & -4 & -24\\
0 & 0 & 1
\end{matrix}
\right]
&=
\left[
\begin{matrix}
s_x cos(\theta) & - s_y sin(\theta) & s_x cos(\theta) t_x - s_y sin(\theta) t_y\\
s_x sin(\theta) & s_y cos(\theta) & s_x sin(\theta) t_x + s_y cos(\theta) t_y\\
0 & 0 & 1
\end{matrix}
\right]\\
s_x cos(\theta) &= 8\\
s_x sin(\theta) &= 6\\
- s_y sin(\theta) &= 3\\
s_y cos(\theta) &= -4\\
s_x cos(\theta) t_x - s_y sin(\theta) t_y &= -7\\
s_x sin(\theta) t_x + s_y cos(\theta) t_y &= -24
\end{align*}

Solving these system of equations, we get a rotation of $\theta = tan^{-1}(\frac{3}{4})$, a scale of $s_x=10, s_y=-5$, and a translation of $t_x=-2, t_y=3$. However, the scaling values are not unique. The steps to solving these equations are left out to keep this answer short.
%----------------------------------------------------------------------------------------
\end{document}
