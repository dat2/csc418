%%%%%%%%%%%%%%%%%%%%%%%%%%%%%%%%%%%%%%%%%
% Short Sectioned Assignment
% LaTeX Template
% Version 1.0 (5/5/12)
%
% This template has been downloaded from:
% http://www.LaTeXTemplates.com
%
% Original author:
% Frits Wenneker (http://www.howtotex.com)
%
% License:
% CC BY-NC-SA 3.0 (http://creativecommons.org/licenses/by-nc-sa/3.0/)
%
%%%%%%%%%%%%%%%%%%%%%%%%%%%%%%%%%%%%%%%%%

%----------------------------------------------------------------------------------------
%	PACKAGES AND OTHER DOCUMENT CONFIGURATIONS
%----------------------------------------------------------------------------------------

\documentclass{article} % A4 paper and 11pt font size

\usepackage[T1]{fontenc} % Use 8-bit encoding that has 256 glyphs
\usepackage[english]{babel} % English language/hyphenation
\usepackage{amsmath} % Math packages
\usepackage{enumerate}
\usepackage{algpseudocode}

\usepackage{fancyhdr} % Custom headers and footers
\pagestyle{fancyplain} % Makes all pages in the document conform to the custom headers and footers
\fancyhead{} % No page header - if you want one, create it in the same way as the footers below
\fancyfoot[L]{} % Empty left footer
\fancyfoot[C]{} % Empty center footer
\fancyfoot[R]{\thepage} % Page numbering for right footer
\setlength{\headheight}{13.6pt} % Customize the height of the header
\renewcommand{\headrulewidth}{0pt} % Remove header underlines
\renewcommand{\footrulewidth}{0pt} % Remove footer underlines

\allowdisplaybreaks

% Margins
\topmargin=-0.8in
\evensidemargin=0in
\oddsidemargin=0in
\textwidth=6.5in
\textheight=9.0in
\headsep=0.5in

%----------------------------------------------------------------------------------------
%	TITLE SECTION
%----------------------------------------------------------------------------------------

\newcommand{\horrule}[1]{\rule{\linewidth}{#1}} % Create horizontal rule command with 1 argument of height

\title{
\normalfont \normalsize
\textsc{University of Toronto} \\ [25pt] % Your university, school and/or department name(s)
\horrule{0.5pt} \\[0.4cm] % Thin top horizontal rule
\huge CSC418 Assignment 1 Part A \\ % The assignment title
\horrule{2pt} \\[0.5cm] % Thick bottom horizontal rule
}

\author{Nicholas Dujay\\999194900} % Your name

\date{\normalsize\today} % Today's date or a custom date

\begin{document}

\maketitle % Print the title

%----------------------------------------------------------------------------------------
%	PROBLEM 1
%----------------------------------------------------------------------------------------

\section{Question 1}
Given that $\vec{e} = (1,2,2)$ $\vec{g} =(1,1,3)$ and $\vec{r}=(0,1,0)$, $\vec{s}$ $\vec{u}$ and $\vec{v}$ are calculated like this:

\begin{align*}
\vec{s}&=-\frac{\vec{g}}{\| \vec{g} \|} = \left(-\frac{1}{\sqrt{11}},-\frac{1}{\sqrt{11}},-\frac{3}{\sqrt{11}} \right) \\
\vec{u}&=\frac{\vec{r} \times \vec{s}}{\|\vec{r} \times \vec{s}\|} = \left(-\frac{1}{\sqrt{5}},0,\frac{1}{\sqrt{5}} \right)\\
\vec{v}&=\frac{\vec{s} \times \vec{u}}{\|\vec{s} \times \vec{u}\|} = \left(-\frac{1}{3\sqrt{2}},-\frac{4}{3\sqrt{2}},-\frac{1}{3\sqrt{2}} \right) 
\end{align*}

From here, we can calculate the world to camera transformation matrix as follows:
\begin{align*}
M_{wc} &= \left[
\begin{array}{ccc|c}
-\frac{1}{\sqrt{5}} & 0 & \frac{1}{\sqrt{5}} & -\frac{1}{\sqrt{5}}\\
\frac{1}{3\sqrt{2}} & -\frac{4}{3\sqrt{2}} & -\frac{1}{3\sqrt{2}} & \frac{11}{3\sqrt{2}}\\
-\frac{1}{\sqrt{11}} & -\frac{1}{\sqrt{11}} & -\frac{3}{\sqrt{11}} & \frac{9}{\sqrt{11}}\\
0 & 0 & 0 & 1
\end{array}
\right]\\
&\text{Where the $M_{wc}$ matrix has the following form:}\\
M_{wc} &= \left[
\begin{array}{ccc|c}
u_x & u_y & u_z & \vec{u} \cdot \vec{e} \\
v_x & v_y & v_z & \vec{v} \cdot \vec{e} \\
s_x & s_y & s_z & \vec{s} \cdot \vec{e} \\
0 & 0 & 0 & 1
\end{array}
\right]
\end{align*}

%----------------------------------------------------------------------------------------
%	PROBLEM 2
%----------------------------------------------------------------------------------------
\newpage
\section{Question 2}
First lets calculate $m$ then project it to $m'$.
\begin{align*}
m &= \frac{1}{2} \left(p+q\right)=\frac{1}{2}\left(
\left[
\begin{matrix}
p_x\\
p_y\\
p_z\\
1
\end{matrix}
\right]
+
\left[
\begin{matrix}
q_x\\
q_y\\
q_z\\
1
\end{matrix}
\right]
\right)
= \left[\begin{matrix}
\frac{1}{2}(p_x + q_x)\\
\frac{1}{2}(p_y + q_y)\\
\frac{1}{2}(p_z + q_z)\\
1
\end{matrix}
\right]\\
m' &= 
\left[
\begin{matrix}
1 & 0 & 0 & 0\\
0 & 1 & 0 & 0\\
0 & 0 & 1 & 0\\
0 & 0 & -\frac{1}{f} & 0\\
\end{matrix}
\right]
\cdot
\left[\begin{matrix}
\frac{1}{2}(p_x + q_x)\\
\frac{1}{2}(p_y + q_y)\\
\frac{1}{2}(p_z + q_z)\\
1
\end{matrix}
\right]
\cong \left[\begin{matrix}
-f\left(\frac{p_x + q_x}{p_z + q_z}\right)\\
-f\left(\frac{p_y + q_y}{p_z + q_z}\right)\\
-f\\
1
\end{matrix}
\right]
\end{align*}


Next lets calculate $p'$ and $q'$ then calculate the midpoint between them.
\begin{align*}
p' &=
\left[
\begin{matrix}
1 & 0 & 0 & 0\\
0 & 1 & 0 & 0\\
0 & 0 & 1 & 0\\
0 & 0 & -\frac{1}{f} & 0\\
\end{matrix}
\right]
\cdot
\left[\begin{matrix}
p_x\\
p_y\\
p_z\\
1
\end{matrix}
\right]
\cong
\left[\begin{matrix}
-f\frac{p_x}{p_z}\\
-f\frac{p_y}{p_z}\\
-f\\
1
\end{matrix}
\right]\\
&\text{$q'$ is calculated similarly.}\\
\frac{1}{2}\left(p'+q'\right) &= 
\frac{1}{2}
\left(
\left[\begin{matrix}
-f\frac{p_x}{p_z}\\
-f\frac{p_y}{p_z}\\
-f\\
1
\end{matrix}
\right]
+
\left[\begin{matrix}
-f\frac{q_x}{q_z}\\
-f\frac{q_y}{q_z}\\
-f\\
1
\end{matrix}
\right]
\right)
\cong
\left[\begin{matrix}
-\frac{f}{2}\left(\frac{p_x}{p_z} + \frac{q_x}{q_z}\right)\\
-\frac{f}{2}\left(\frac{p_y}{p_z} + \frac{q_y}{q_z}\right)\\
-f\\
1
\end{matrix}
\right]
\end{align*}
In order for $m' = \frac{1}{2}(p'+q')$, we need $-f\left(\frac{p_x + q_x}{p_z + q_z}\right) = -\frac{f}{2}\left(\frac{p_x}{p_z} + \frac{q_x}{q_z}\right)$ and $-f\left(\frac{p_y + q_y}{p_z + q_z}\right) = -\frac{f}{2}\left(\frac{p_y}{p_z} + \frac{q_y}{q_z}\right)$. There's only one case where this is true, and it is when $p_z = q_z$.
\\
If we instead use an orthographic projection, then $m' = \frac{1}{2}(p'+q')$ for all cases:
\begin{align*}
m' &= 
\left[
\begin{matrix}
1 & 0 & 0 & 0\\
0 & 1 & 0 & 0\\
0 & 0 & 0 & 1\\
\end{matrix}
\right]
\cdot
\left[\begin{matrix}
\frac{1}{2}(p_x + q_x)\\
\frac{1}{2}(p_y + q_y)\\
\frac{1}{2}(p_z + q_z)\\
1
\end{matrix}
\right]
= 
\left[\begin{matrix}
\frac{1}{2}(p_x + q_x)\\
\frac{1}{2}(p_y + q_y)\\
1
\end{matrix}
\right]\\
p' &= 
\left[
\begin{matrix}
1 & 0 & 0 & 0\\
0 & 1 & 0 & 0\\
0 & 0 & 0 & 1\\
\end{matrix}
\right]
\cdot
\left[\begin{matrix}
p_x\\
p_y\\
p_z\\
1
\end{matrix}
\right] = \left[\begin{matrix}
p_x\\
p_y\\
1
\end{matrix}
\right]\\
\frac{1}{2}\left(p' + q'\right) &=
\frac{1}{2}
\left(
\left[\begin{matrix}
p_x\\
p_y\\
1
\end{matrix}
\right]
+
\left[\begin{matrix}
q_x\\
q_y\\
1
\end{matrix}
\right]
\right)\\
&=
\left[\begin{matrix}
\frac{1}{2}(p_x + q_x)\\
\frac{1}{2}(p_y + q_y)\\
1
\end{matrix}
\right]= m'
\end{align*}

%----------------------------------------------------------------------------------------
%	PROBLEM 3
%----------------------------------------------------------------------------------------
\newpage
\section{Question 3}

%----------------------------------------------------------------------------------------

%----------------------------------------------------------------------------------------
%	PROBLEM 4
%----------------------------------------------------------------------------------------

\newpage
\section{Question 4}
%----------------------------------------------------------------------------------------

%----------------------------------------------------------------------------------------
%	PROBLEM 5
%----------------------------------------------------------------------------------------

\newpage
\section{Question 5}
%----------------------------------------------------------------------------------------

%----------------------------------------------------------------------------------------
%	PROBLEM 6
%----------------------------------------------------------------------------------------

\newpage
\section{Question 6}
%----------------------------------------------------------------------------------------
\end{document}
