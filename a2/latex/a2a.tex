%%%%%%%%%%%%%%%%%%%%%%%%%%%%%%%%%%%%%%%%%
% Short Sectioned Assignment
% LaTeX Template
% Version 1.0 (5/5/12)
%
% This template has been downloaded from:
% http://www.LaTeXTemplates.com
%
% Original author:
% Frits Wenneker (http://www.howtotex.com)
%
% License:
% CC BY-NC-SA 3.0 (http://creativecommons.org/licenses/by-nc-sa/3.0/)
%
%%%%%%%%%%%%%%%%%%%%%%%%%%%%%%%%%%%%%%%%%

%----------------------------------------------------------------------------------------
%	PACKAGES AND OTHER DOCUMENT CONFIGURATIONS
%----------------------------------------------------------------------------------------

\documentclass{article} % A4 paper and 11pt font size

\usepackage[T1]{fontenc} % Use 8-bit encoding that has 256 glyphs
\usepackage[english]{babel} % English language/hyphenation
\usepackage{amsmath} % Math packages
\usepackage{enumerate}
\usepackage{algpseudocode}

\usepackage{fancyhdr} % Custom headers and footers
\pagestyle{fancyplain} % Makes all pages in the document conform to the custom headers and footers
\fancyhead{} % No page header - if you want one, create it in the same way as the footers below
\fancyfoot[L]{} % Empty left footer
\fancyfoot[C]{} % Empty center footer
\fancyfoot[R]{\thepage} % Page numbering for right footer
\setlength{\headheight}{13.6pt} % Customize the height of the header
\renewcommand{\headrulewidth}{0pt} % Remove header underlines
\renewcommand{\footrulewidth}{0pt} % Remove footer underlines

\allowdisplaybreaks

% Margins
\topmargin=-0.8in
\evensidemargin=0in
\oddsidemargin=0in
\textwidth=6.5in
\textheight=9.0in
\headsep=0.5in

%----------------------------------------------------------------------------------------
%	TITLE SECTION
%----------------------------------------------------------------------------------------

\newcommand{\horrule}[1]{\rule{\linewidth}{#1}} % Create horizontal rule command with 1 argument of height

\title{
\normalfont \normalsize
\textsc{University of Toronto} \\ [25pt] % Your university, school and/or department name(s)
\horrule{0.5pt} \\[0.4cm] % Thin top horizontal rule
\huge CSC418 Assignment 1 Part A \\ % The assignment title
\horrule{2pt} \\[0.5cm] % Thick bottom horizontal rule
}

\author{Nicholas Dujay\\999194900} % Your name

\date{\normalsize\today} % Today's date or a custom date

\begin{document}

\maketitle % Print the title

%----------------------------------------------------------------------------------------
%	PROBLEM 1
%----------------------------------------------------------------------------------------

\section{Question 1}
Given that $\vec{e} = (1,2,2)$ $\vec{g} =(1,1,3)$ and $\vec{r}=(0,1,0)$, $\vec{s}$ $\vec{u}$ and $\vec{v}$ are calculated like this:

\begin{align*}
\vec{s}&=-\frac{\vec{g}}{\| \vec{g} \|} = \left(-\frac{1}{\sqrt{11}},-\frac{1}{\sqrt{11}},-\frac{3}{\sqrt{11}} \right) \\
\vec{u}&=\frac{\vec{r} \times \vec{s}}{\|\vec{r} \times \vec{s}\|} = \left(-\frac{1}{\sqrt{5}},0,\frac{1}{\sqrt{5}} \right)\\
\vec{v}&=\frac{\vec{s} \times \vec{u}}{\|\vec{s} \times \vec{u}\|} = \left(-\frac{1}{3\sqrt{2}},-\frac{4}{3\sqrt{2}},-\frac{1}{3\sqrt{2}} \right) 
\end{align*}

From here, we can calculate the world to camera transformation matrix as follows:
\begin{align*}
M_{wc} &= \left[
\begin{array}{ccc|c}
-\frac{1}{\sqrt{5}} & 0 & \frac{1}{\sqrt{5}} & -\frac{1}{\sqrt{5}}\\
\frac{1}{3\sqrt{2}} & -\frac{4}{3\sqrt{2}} & -\frac{1}{3\sqrt{2}} & \frac{11}{3\sqrt{2}}\\
-\frac{1}{\sqrt{11}} & -\frac{1}{\sqrt{11}} & -\frac{3}{\sqrt{11}} & \frac{9}{\sqrt{11}}\\
0 & 0 & 0 & 1
\end{array}
\right]\\
&\text{Where the $M_{wc}$ matrix has the following form:}\\
M_{wc} &= \left[
\begin{array}{ccc|c}
u_x & u_y & u_z & \vec{u} \cdot \vec{e} \\
v_x & v_y & v_z & \vec{v} \cdot \vec{e} \\
s_x & s_y & s_z & \vec{s} \cdot \vec{e} \\
0 & 0 & 0 & 1
\end{array}
\right]
\end{align*}

%----------------------------------------------------------------------------------------
%	PROBLEM 2
%----------------------------------------------------------------------------------------

\section{Question 2}
First lets calculate $m$ then project it to $m'$.
\begin{align*}
m &= \frac{1}{2} \left(p+q\right)=\frac{1}{2}\left(
\left[
\begin{matrix}
p_x\\
p_y\\
p_z\\
1
\end{matrix}
\right]
+
\left[
\begin{matrix}
q_x\\
q_y\\
q_z\\
1
\end{matrix}
\right]
\right)
= \left[\begin{matrix}
\frac{1}{2}(p_x + q_x)\\
\frac{1}{2}(p_y + q_y)\\
\frac{1}{2}(p_z + q_z)\\
1
\end{matrix}
\right]\\
m' &= 
\left[
\begin{matrix}
1 & 0 & 0 & 0\\
0 & 1 & 0 & 0\\
0 & 0 & 1 & 0\\
0 & 0 & -f & 0\\
\end{matrix}
\right]
\cdot
\left[\begin{matrix}
\frac{1}{2}(p_x + q_x)\\
\frac{1}{2}(p_y + q_y)\\
\frac{1}{2}(p_z + q_z)\\
1
\end{matrix}
\right]
\cong \left[\begin{matrix}
-\frac{1}{f}\left(\frac{p_x + q_x}{p_z + q_z}\right)\\
-\frac{1}{f}\left(\frac{p_y + q_y}{p_z + q_z}\right)\\
-\frac{1}{f}\\
1
\end{matrix}
\right]
\end{align*}


Next lets calculate $p'$ and $q'$ then calculate the midpoint between them.
\begin{align*}
p' &=
\left[
\begin{matrix}
1 & 0 & 0 & 0\\
0 & 1 & 0 & 0\\
0 & 0 & 1 & 0\\
0 & 0 & -f & 0\\
\end{matrix}
\right]
\cdot
\left[\begin{matrix}
p_x\\
p_y\\
p_z\\
1
\end{matrix}
\right]
\cong
\left[\begin{matrix}
-\frac{p_x}{p_z f}\\
-\frac{p_y}{p_z f}\\
-\frac{1}{f}\\
1
\end{matrix}
\right]\\
&\text{$q'$ is calculated similarly.}\\
\frac{1}{2}\left(p'+q'\right) &= 
\frac{1}{2}
\left(
\left[\begin{matrix}
-\frac{p_x}{p_z f}\\
-\frac{p_y}{p_z f}\\
-\frac{1}{f}\\
1
\end{matrix}
\right]
+
\left[\begin{matrix}
-\frac{q_x}{q_z f}\\
-\frac{q_y}{q_z f}\\
-\frac{1}{f}\\
1
\end{matrix}
\right]
\right)
\cong
\left[\begin{matrix}
-\frac{1}{2f}\left(\frac{p_x}{p_z} + \frac{q_x}{q_z}\right)\\
-\frac{1}{2f}\left(\frac{p_y}{p_z} + \frac{q_y}{q_z}\right)\\
-\frac{1}{f}\\
1
\end{matrix}
\right]
\end{align*}

Therefore, $m' \neq \frac{1}{2}(p'+q')$ for this projection. However, if we use an orthographic projection, then $m' = \frac{1}{2}(p'+q')$:

\begin{align*}
m' &= 
\left[
\begin{matrix}
1 & 0 & 0 & 0\\
0 & 1 & 0 & 0\\
0 & 0 & 0 & 1\\
\end{matrix}
\right]
\cdot
\left[\begin{matrix}
\frac{1}{2}(p_x + q_x)\\
\frac{1}{2}(p_y + q_y)\\
\frac{1}{2}(p_z + q_z)\\
1
\end{matrix}
\right]
= 
\left[\begin{matrix}
\frac{1}{2}(p_x + q_x)\\
\frac{1}{2}(p_y + q_y)\\
1
\end{matrix}
\right]\\
p' &= 
\left[
\begin{matrix}
1 & 0 & 0 & 0\\
0 & 1 & 0 & 0\\
0 & 0 & 0 & 1\\
\end{matrix}
\right]
\cdot
\left[\begin{matrix}
p_x\\
p_y\\
p_z\\
1
\end{matrix}
\right] = \left[\begin{matrix}
p_x\\
p_y\\
1
\end{matrix}
\right]\\
\frac{1}{2}\left(p' + q'\right) &=
\frac{1}{2}
\left(
\left[\begin{matrix}
p_x\\
p_y\\
1
\end{matrix}
\right]
+
\left[\begin{matrix}
q_x\\
q_y\\
1
\end{matrix}
\right]
\right)\\
&=
\left[\begin{matrix}
\frac{1}{2}(p_x + q_x)\\
\frac{1}{2}(p_y + q_y)\\
1
\end{matrix}
\right]= m'
\end{align*}

%----------------------------------------------------------------------------------------
%	PROBLEM 3
%----------------------------------------------------------------------------------------

\section{Question 3}

\begin{enumerate}[a)]

%----------------------------------------------------------------------------------------
% part a
%----------------------------------------------------------------------------------------
\item

First, we get the direction vector from $\vec{p}_i$ to $\vec{p}_{i+1}$: $(x_{i+1} - x_i, y_{i+1} - y_i)$. Then, since the inward facing normal is a 90 degree counter clockwise rotation of this vector, the result is $(y_i - y_{i+1}, x_{i+1} - x_i)$.

Counter clockwise rotation:
\begin{align*}
\left[
\begin{matrix}
cos(90) & -sin(90)\\
sin(90) & cos(90)
\end{matrix}
\right]
\cdot
\left[
\begin{matrix}
x\\
y
\end{matrix}
\right] = \left[
\begin{matrix}
0 & -1\\
1 & 0
\end{matrix}
\right]
\cdot
\left[
\begin{matrix}
x\\
y
\end{matrix}
\right] = \left[
\begin{matrix}
-y\\
x
\end{matrix}
\right]
\end{align*}
%----------------------------------------------------------------------------------------

%----------------------------------------------------------------------------------------
% part b
%----------------------------------------------------------------------------------------
\item
Let $d_q$ be a vector from $q$ to $p_i$, ie $d_q = p_i - q$. The following function tells you if the point is on the inward facing line:
\[
sameside(q, p_i, n_i) =
\begin{cases}
true & \text{if } d_q \cdot n_i > 0\\
false & \text{if } d_q \cdot n_i \leq 0
\end{cases}
\]

If the angle between $d_q$ and $n_i$ is less than 90 degrees, we know $q$ is on the inward side. Since $\forall \theta \in [0,90) \cdot cos(\theta) > 0$, then the dot product between $d_q$ and $n_i$ is positive if $q$ is on the inward side, and negative if $q$ is on the outward side.
%----------------------------------------------------------------------------------------

%----------------------------------------------------------------------------------------
% part c
%----------------------------------------------------------------------------------------
\item

The algorithm will use the above two parts to the question to solve this. The high level steps are the following:

\begin{enumerate}[i)]
\item For each pair of points $(p_i, p_{i+1})$ calculate the \textbf{inward facing normal} $n_{p_i}$.
\item For each pair of points $(r_i, r_{i+1})$ calculate the \textbf{outward facing normal} $n_{r_i}$. Note: in the actual implementation, we can just reverse the order of points $(r_{i+1}, r_i)$ and calculate the inward facing normal.
\item For all of the above normals, ensure that \textbf{sameside} (from part b) is true for the point.
\end{enumerate}

Some pseudo code:
\begin{algorithmic}
\Function{normal}{$p_{i+1}$, $p_i$}
\State \Return ($y_i - y_{i+1}$, $x_{i+1} - x_i$)
\EndFunction
\State
\Function{InConvexPolgyon}{point}
\ForAll{$p_i, p_{i+1}$ in the outer polygon} \Comment{check if the point is inside the outer polygon}
\If{not sameside(point, $p_i$, \Call{Normal}{$p_{i+1}$,$p_i$})}
\State \Return False
\EndIf
\EndFor

\ForAll{$r_j, r_{j+1}$ in the inner polygon} \Comment{check if the point is outside the inner polygon}
\If{not sameside(point, $r_i$, \Call{Normal}{$r_j$,$r_{j+1}$})}
\State \Return False \Comment{reversing the order gives you an outward facing normal here}
\EndIf
\EndFor
\State \Return True
\EndFunction
\end{algorithmic}

%----------------------------------------------------------------------------------------

\end{enumerate}

%----------------------------------------------------------------------------------------

%----------------------------------------------------------------------------------------
%	PROBLEM 4
%----------------------------------------------------------------------------------------

\section{Question 4}

To calculate the Homography H, we start with the following matrix equation:\\
\begin{align*}
\left[
\begin{matrix}
a & b & c\\
d & e & f\\
h & k & 1
\end{matrix}
\right]
\cdot
\left[
\begin{matrix}
x_k\\
y_k\\
1
\end{matrix}
\right]
&\cong
\left[
\begin{matrix}
u_k\\
v_k\\
1
\end{matrix}
\right]\\
\left[
\begin{matrix}
a x_k + b y_k + c\\
d x_k + e y_k + f\\
h x_k + k y_k + 1
\end{matrix}
\right]
&\cong
\left[
\begin{matrix}
u_k\\
v_k\\
1
\end{matrix}
\right]
\end{align*}

Since these two vectors are homogeneously equivalent, we divide through by the z coordinate to get two equations:\\
\begin{align*}
\left[
\begin{matrix}
\frac{a x_k + b y_k + c}{h x_k + k y_k + 1}\\
\frac{d x_k + e y_k + f}{h x_k + k y_k + 1}\\
1
\end{matrix}
\right]
&=
\left[
\begin{matrix}
u_k\\
v_k\\
1
\end{matrix}
\right]\\
a x_k + b y_k + c - u_k (h x_k + k y_k + 1) &= 0\\
d x_k + e y_k + f - v_k (h x_k + k y_k + 1) &= 0
\end{align*}

From these two equations, we just plug in $x_k, y_k, u_k, v_k$ and get the following system of equations:
\begin{align*}
a 0 + b 0 + c + 4 (h 0 + k 0 + 1) &= 0\\
d 0 + e 0 + f - 1 (h 0 + k 0 + 1) &= 0\\
a 1 + b 0 + c + 1.5 (h 1 + k 0 + 1) &= 0 \\
d 1 + e 0 + f + 0.5 (h 1 + k 0 + 1) &= 0\\
a 0 + b 1 + c - 0.5 (h 0 + k 1 + 1) &= 0\\
d 0 + e 1 + f + 0.5 (h 0 + k 1 + 1) &= 0\\
a 1 + b 1 + c - 0 (h 1 + k 1 + 1) &= 0\\
d 1 + e 1 + f - 1 (h 1 + k 1 + 1) &= 0
\end{align*}
Solving all of these equations, we get $a = 1, b = 3, c = -4, d = -2, e = 0, f = 1, h=1, k=-3$.
%----------------------------------------------------------------------------------------

%----------------------------------------------------------------------------------------
%	PROBLEM 5
%----------------------------------------------------------------------------------------

\section{Question 5}

First, lets express the transformation as a translation and then followed by a scale and finally followed by a rotation. Let the given matrix $ \left[
\begin{matrix}
8 & 3 & -7\\
6 & -4 & -24\\
0 & 0 & 1
\end{matrix}
\right]$ be $A$,
Let $R(\theta)$ be the rotation matrix, $S(s_x, s_y)$ be the non uniform scale matrix, $T(t_x, t_y)$ be the translation matrix.
\begin{align*}
R(\theta) \cdot S(s_x, s_y) \cdot T(t_x, t_y) &=
R(\theta) \cdot S(s_x, s_y) \cdot
\left[
\begin{matrix}
1 & 0 & t_x\\
0 & 1 & t_y\\
0 & 0 & 1
\end{matrix}
\right]\\
&=
R(\theta) \cdot
\left[
\begin{matrix}
s_x & 0 & 0\\
0 & s_y & 0\\
0 & 0 & 1
\end{matrix}
\right]
\cdot
\left[
\begin{matrix}
1 & 0 & t_x\\
0 & 1 & t_y\\
0 & 0 & 1
\end{matrix}
\right]\\
&=
R(\theta) \cdot
\left[
\begin{matrix}
s_x & 0 & s_x t_x\\
0 & s_y & s_y t_y\\
0 & 0 & 1
\end{matrix}
\right]\\
&= \left[
\begin{matrix}
cos(\theta) & -sin(\theta) & 0\\
sin(\theta) & cos(\theta) & 0\\
0 & 0 & 1
\end{matrix}
\right]
\cdot
\left[
\begin{matrix}
s_x & 0 & s_x t_x\\
0 & s_y & s_y t_y\\
0 & 0 & 1
\end{matrix}
\right]\\
&= \left[
\begin{matrix}
s_x cos(\theta) & - s_y sin(\theta) & s_x cos(\theta) t_x - s_y sin(\theta) t_y\\
s_x sin(\theta) & s_y cos(\theta) & s_x sin(\theta) t_x + s_y cos(\theta) t_y\\
0 & 0 & 1
\end{matrix}
\right]
\end{align*}

Setting $A = R(\theta) \cdot S(s_x, s_y) \cdot T(t_x, t_y)$ will give us the following system of equations.
\begin{align*}
\left[
\begin{matrix}
8 & 3 & -7\\
6 & -4 & -24\\
0 & 0 & 1
\end{matrix}
\right]
&=
\left[
\begin{matrix}
s_x cos(\theta) & - s_y sin(\theta) & s_x cos(\theta) t_x - s_y sin(\theta) t_y\\
s_x sin(\theta) & s_y cos(\theta) & s_x sin(\theta) t_x + s_y cos(\theta) t_y\\
0 & 0 & 1
\end{matrix}
\right]\\
s_x cos(\theta) &= 8\\
s_x sin(\theta) &= 6\\
- s_y sin(\theta) &= 3\\
s_y cos(\theta) &= -4\\
s_x cos(\theta) t_x - s_y sin(\theta) t_y &= -7\\
s_x sin(\theta) t_x + s_y cos(\theta) t_y &= -24
\end{align*}

Solving these system of equations, we get a rotation of $\theta = tan^{-1}(\frac{3}{4})$, a scale of $s_x=10, s_y=-5$, and a translation of $t_x=-2, t_y=3$. The steps to solving these equations are left out to keep this answer short.
%----------------------------------------------------------------------------------------
\end{document}
